%!TEX root = cv.tex

% These two lengths set the left margin and the space reserved for the sectional titles, which are, by design *not* part of the text block.  If you need to invade that space for a special section, you just need to decrease the left margin by '-\sectionlen' (e.g., \hspace*{-\sectionlen})
\newlength\lmargwidth
\setlength{\lmargwidth}{7ex}  % width left of section headings (this is also the right margin)

\newlength\sectionlen
\setlength{\sectionlen}{1.1in}  % space reserved for the section headings

\newlength\itemspacelen
\setlength{\itemspacelen}{2ex}  % vertical space between items in same section
\newcommand*{\itemspace}{\vspace{\itemspacelen}}

\newlength\secspacelen
\setlength{\secspacelen}{3.5ex}  % vertical space between sections; used with \reallynopagebreak

\newcommand*{\subsectionone}[1]{\textbf{#1}}

% Set up \section and \subsection format
\usepackage{titlesec}
\setcounter{secnumdepth}{0}  % don't number sections or subsections

\titleformat{\section}[leftmargin]
   {\small\scshape\filright}
   {\thesection}
   {1em}  % horizontal separation between label (numbering) and title
   {}

\titlespacing*{\section}{\sectionlen}{\secspacelen}{0ex}  % * == no indentation for 1st line

\titleformat{\subsection}[runin]  % \titleformat{command}[shape]{format}{label}{sep}
   {\bfseries}
   {\thesubsection}
   {1em}
   {}

\titlespacing*{\subsection}  % * == no indentation for 1st line
  {0em}{\itemspacelen}{1ex}

\usepackage{enumitem}  % for noitemsep and nolistsep
\setlist{nosep,leftmargin=1.5em,label=$\bullet$}  % label=$\bullet$: larger bullets

% Margins: use geometry package to set margins, and refrain from setting them in fancyhdr
\usepackage[paper=letterpaper,
	lmargin={\dimexpr\lmargwidth +\sectionlen\relax},
	rmargin=\lmargwidth,
	vmargin={5ex,5ex},  % top and bottom gaps between boundaries of physical paper and text body area
	headsep=0ex,  % gap between top of text body area and bottom of header text (not decorative header line)
	headheight=3ex,  % gap between bottom of header text and top of header space; if too small, header text is pushed below the intended bottom of header text
	footskip=3ex,  % gap between bottom of text body area and bottom of footer text (where page number is written); if too small, footer text is pushed below the intended bottom of footer text
	nomarginpar,
	% showframe  % for debugging
	]{geometry}
	

% Hearder and footer
\usepackage[nodayofweek]{datetime}
\newdateformat{mydateformat}{\monthname[\THEMONTH] \THEYEAR}

\usepackage{lastpage}

\usepackage{fancyhdr}
\pagestyle{fancy}
\fancyhf{}
\fancyheadoffset[L]{\sectionlen}  % start header text from left column
\fancyfootoffset[L]{\sectionlen}  % start footer text from left column, thereby centering page numbers
\lhead{ \fancyplain{}{\textbf{\myname/}} }
\rhead{ \fancyplain{}{As of \mydateformat\mydate/} }  % for specific date (e.g., September 2015), \rhead{ \fancyplain{}{As of September 2015} }
\cfoot{ \fancyplain{}{\thepage/\pageref*{LastPage}} }

\fancypagestyle{firstpagestyle}
{
	\newgeometry{  % increase top margin and headsep of 1st page to align its geometry with 2nd page despite increased header font
		lmargin={\dimexpr\lmargwidth +\sectionlen\relax},
		rmargin=\lmargwidth,
		vmargin={5.5ex,5ex},  % increase top margin to align page 1 and 2's first lines of text
		headsep=1.7ex,  % increase headsep to align page 1 and 2's decorative header lines
		headheight=3ex,
		footskip=3ex,
		nomarginpar
	}	
	\fancyhf{}
	\fancyheadoffset[L]{\sectionlen}  % start header text from left column
	\fancyfootoffset[L]{\sectionlen}  % start footer text from left column, thereby centering page numbers
	\lhead{ \fancyplain{}{\Large\textbf{\myname/}} }
	\rhead{ \fancyplain{}{\large{As of \mydateformat\mydate/}} }
	\cfoot{ \fancyplain{}{\thepage/\pageref*{LastPage}} }
}
